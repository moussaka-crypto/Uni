Das Spiel PONG ermöglicht es dem \gls{spieler} sich im Spiel zu verbessern, mit anderen zu konkurrieren, und bei Gelegenheit der Realität zu entfliehen.
\\
In der App kann der \gls{spieler} \glspl{skin} freischalten, wodurch der Spielspaß erhalten bleibt und der \gls{spieler} eine Langzeitmotivation erhält. Außerdem gibt es eine Monetarisierung durch Werbung.

Durch die kostenfreie Bereitstellung der App erhofft sich \hyperref[sec:auftraggeber]{der Auftraggeber} eine großflächige
Ausbreitung innerhalb der gewählten Zielgruppe. Dabei liegt der langfristige Nutzen hauptsächlich im Ertrag durch Werbeeinnahmen.
Diese fallen an, sofern der \gls{spieler} sich dazu entschließt ein Werbevideo anzuschauen, um nach dem ersten Verlieren aller \gls{leben} weiterspielen zu können (siehe dazu Aktivitätsdiagramm \hyperref[fig:dia:ads]{u12}, Game-Over Menü \hyperref[subsec:u11-gameOverMenu]{u11}).
