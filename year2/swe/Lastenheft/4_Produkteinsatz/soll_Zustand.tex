Der letztendliche SOLL-Zustand ergibt sich final aus den \hyperref[sec:requirements]{Requirements} (Kapitel \ref{sec:requirements}).
Im Allgemeinen lässt sich die Applikation wie folgt zusammenfassen:

\begin{itemize}
    \item Die App muss aus einem Startbildschirm, Spiel-, Store-, \gls{Top10} und Credits-Screen bestehen.
    \item Vom Startbildschirm, müssen Spiel, Store, \gls{Top10} und Credits durch Buttons erreichbar sein.
    \item Auf jedem Screen muss es einen Zurück-Button geben.
    \item Das Spiel muss Singleplayer und Hochkant sein.
    \item Es \gls{sollte} aus zwei Spielmodi: \gls{classicMode} und \gls{invasionMode} ausgewählt werden können.
    \item Classic und Invasion sollen aus den klassischen Elementen: Ball und Balken bestehen.
    \item Im Invasion Modus sollen die Kästchen eine bestimmte Anzahl von Malen getroffen werden, um zerstört zu werden.
    \item Pro Spielmodus muss es drei verschiedenen Schwierigkeitsstufen geben: "Easy", "Medium" und "Hard".
    \item Des Weiteren, muss es \glspl{powerup} geben.
    \item Unabhängig von Modus muss das Spiel jederzeit durch einen Button pausierbar sein.
    \item Sobald der Ball unterhalb des Balkens ist, muss der Spieler ein Leben verlieren.
    \item Wenn alle drei Leben aufgebraucht sind, muss es eine einmalige Chance geben, ein weiteres Leben durch das Anschauen einer Werbung zu bekommen.
    \item Falls der Spieler einen neuen Highscore erreicht hat, muss es nach dem Spiel die Möglichkeit geben seinen Namen einzutragen.
    \item Nach jedem Spiel muss dem Spieler sein Score und die verdienten Coins angezeigt werden.
    \item Im Store muss man mit Coins, Skins für Ball, Ball-Schweif, Balken und Hintergrund kaufen und auswählen können.
    \item Der Top-10-Screen muss die Highscores, Namen, Spielmodi und Schwie\-rig\-keits\-stufen anzeigen.
    \item Im Credits-Screen müssen die Namen von Personen, die an der Entwicklung der App gearbeitet haben und das Copyright, angezeigt werden.
\end{itemize}
