Einige Anforderungen lassen sich nicht genau spezifizieren und/oder können erst während der Entwicklung genauer festgelegt werden.
Dies umfasst unter anderem - aber nicht beschränkend auf - das Reaktionsverhalten und die Haptik der \gls{balken}-Steuerung,
die Gewichtung und Verteilung numerischer Attribute wie z.B. \gls{point}-Multiplikatoren je Schwierigkeitsgrad und einige Designs.

\vspace{1em}

Diese Variablen werden absichtlich nicht in diesem Lastenheft aufgeführt, da die letztendliche Festsetzung dieser Parameter
"auf gut Glück" das Design der Applikation auf eine festgelegte Richtung setzt, von der zum aktuellen Stand nicht
absehbar ist, ob sie dem Auftraggeber gefällt.

\vspace{1em}

Die Parteien einigen sich auf eine offene Formulierung dieser Anforderungen, wie in Kapitel \ref{sec:requirements} und
ein Balancing im Laufe der Spielentwicklung. Der Auftraggeber erhält durch die \gls{beta} die Möglichkeit, weitere Änderungswünsche
bezüglich dieser Parameter zu äußern.

Die maximale Anzahl an Änderungsiterationen zwischen \gls{beta} und \gls{release} wird hiermit auf 3 beschränkt.
