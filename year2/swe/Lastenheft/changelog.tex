Die folgenden Änderungen wurden im Laufe der Entwicklung des Lastenhefts vollzogen. Diese erfolgten aus den Gesprächen und dem Schriftverkehr zwischen Kunden (siehe \hyperref[sec:auftraggeber]{Auftraggeber}) und dem Vertrieb (Hristomir Dimov \& Leon Wiesen).

\begin{xltabular}{\textwidth}{|c|X|}
    \hline
    \textbf{Version}   & \textbf{Änderungen}     \\
    \hline

    v0.0  \textit{(27.10.2022)} &  \begin{itemize}
        \item Kap. \ref{sec:funktionen}: Hinzugefügt
        \item Kap. \ref{sec:requirements}: Hinzugefügt
    \end{itemize}
    \\ \hline

    v0.1  \textit{(03.11.2022)} &  \begin{itemize}
        \item Kap. \ref{sec:auftraggeber}: Hinzugefügt
        \item Kap. \ref{sec:zeitbudget}: Hinzugefügt
        \item Kap. \ref{sec:bestimmung}: Hinzugefügt
        \item Kap. \ref{sec:einsatz}: Hinzugefügt
        \item Kap. \ref{sec:funktionen}: Neue MockUps
        \item Kap. \ref{sec:produktdaten}: Hinzugefügt
        \item Kap. \ref{sec:performance}: Hinzugefügt
        \item Kap. \ref{sec:quality}: Hinzugefügt
        \item Kap. \ref{sec:requirements}: Requirements erweitert
    \end{itemize}
    \\ \hline

    v1.0  \textit{(07.11.2022)} & \begin{itemize}
        \item Rechtscheib- \& Grammatikfehler korrigiert
        \item Glossar: Neue Begriffe Hinzugefügt
        \item Deckblatt: Gruppenname vervollständigt
        \item Deckblatt: Build-Datum hinzugefügt
        \item Kap. \ref{sec:auftraggeber}: Spezifischer formuliert
        \item Kap. \ref{sec:zeitbudget}: Glossar-Verlinkungen
        \item Kap. \ref{sec:zeitbudget}: Ausführlichere Grafik
        \item Kap. \ref{sec:einsatz}: Use-Case Diagramm hinzugefügt
        \item Kap. \ref{sec:funktionen}: Neue Diagramme
        \item Kap. \ref{sec:funktionen}: Genauere Beschreibungen
        \item Kap. \ref{sec:funktionen}: Glossar-Verlinkungen
        \item Kap. \ref{sec:produktdaten}: Genauere Beschreibungen
        \item Kap. \ref{subsec:lieferumfang}: Lieferumfang hinzugefügt
        \item Kap. \ref{sec:performance}: Minimierung auf das Nötigste
        \item Kap. \ref{sec:requirements}: Achievements sind jetzt mandatory (\gls{muss})
        \item Abnahme Hinzugefügt
    \end{itemize}
    \\ \hline

    v1.1  \textit{(08.11.2022)} & \begin{itemize}
        \item Rechtscheib- \& Grammatikfehler korrigiert
        \item Korrekte Verlinkung der Kapitel \& Bilder
        \item Vereinheitlichung von Begrifflichkeiten
        \item Vereinheitlichung von Begrifflichkeiten
        \item Glossar: Neue Begriffe Hinzugefügt
        \item Kap. \ref{sec:funktionen}: Korrekte Nummerierung der Bilder
        \item Kap. \ref{sec:funktionen}: Korrektur von Diagrammen
        \item Kap. \ref{subsec:balancing}: Hinzugefügt
    \end{itemize}
    \\ \hline

    v2.0  \textit{(09.11.2022)} & \begin{itemize}
        \item Rechtscheib- \& Grammatikfehler korrigiert
        \item Glossar: Einheitliche Benennung
        \item Glossar: Neue Begriffe hinzugefügt
        \item Zeitbudget vervollständigt
        \item Kap.~\ref{subsec:balancing}: Hinzugefügt
        \item Kap.~\ref{subsec:tests}: Hinzugefügt
        \item Hinweise vereinheitlicht
        \item Kap.~\ref{sec:funktionen}: Use-Case Diagramm angepasst
        \item Falsche automatische Silbentrennung korrigiert
        \item Figure-Benennung in Texten
        \item Figure~\ref{fig:dia:u40}: Platzhalter ausgetauscht gegen Werte
        \item Abnahme Auf \ref{subsec:balancing}, \ref{subsec:lieferumfang}, \ref{subsec:tests} angepasst
    \end{itemize}
    \\ \hline

    v2.1  \textit{(09.11.2022)} & \begin{itemize}
        \item Rechtscheib- \& Grammatikfehler korrigiert
        \item Formatierungswünsche des Auftraggebers
        \item Umformulierungswünsche des Auftraggebers
        \item Falsche Kapitelnummer in Kapitel \ref{subsec:funktionenEinleitung} gefixed
        \item Links in Kapitel \ref{subsec:alleFunktionen} hinzugefügt

    \end{itemize}
    \\ \hline


\end{xltabular}

% 
% 
% 
% 
% 
% 
% 
% 
% 
% 
% 
% 
% 
% 
% 
